% $Log: abstract.tex,v $
% Revision 1.1  93/05/14  14:56:25  starflt
% Initial revision
% 
% Revision 1.1  90/05/04  10:41:01  lwvanels
% Initial revision
% 
%
%% The text of your abstract and nothing else (other than comments) goes here.
%% It will be single-spaced and the rest of the text that is supposed to go on
%% the abstract page will be generated by the abstractpage environment.  This
%% file should be \input (not \include 'd) from cover.tex.

Array-based programming environments are popular for scientific and
technical computing.
These systems consist of built-in function libraries paired with high-level
languages for interaction.
Although the libraries perform well, it is widely believed that scripting in these
languages is necessarily slow, and that only heroic feats of engineering can at
best partially ameliorate this problem.

This thesis  argues that what is really needed is a more coherent
structure for this functionality.
To find one, we must ask what technical computing is really about.
This thesis suggests that this kind of programming is characterized by an emphasis on operator
complexity and code specialization, and that a language can be designed to
better fit these requirements.

The key idea is to integrate code \emph{selection} with code \emph{specialization},
using generic functions and data-flow type inference.
Systems like these can suffer from inefficient compilation, or from
uncertainty about what exactly to specialize on.
We show that dispatch on structured type tags helps address these problems.
The resulting language, Julia, achieves a Quine-style
``explication by elimination'' of many of the productive features
technical computing users expect.



%For this role I propose an abstraction based on an extended version of
%generic functions.
%The novelty of this mechanism is that it is both flexible enough to describe
%the wide variety of behaviors users need in practice, while also providing
%enough information to a compiler to yield good performance.


% integration of selection and specialization

% making data-flow and specialization-based languages practical

% answers the question of what to specialize on

% applies Quine's ``explication through elimination'' to common features of T.C.
