\chapter{The Julia approach}

\section{Core calculus and data model}
\label{sec:corecalc}

Julia is based on an untyped lambda calculus augmented with generic functions,
tagged data values, and mutable cells.

\vspace{-3ex}
\begin{singlespace}
\begin{align*}
  e\ ::=\ &\ x                 & \textrm{(variable)} \\
        &\ |\ 0\ |\ 1\ |\ \cdots     & \textrm{(constant)} \\
        &\ |\ \texttt{new}(e_{tag}, e_1, e_2, \cdots) & \textrm{(data constructor)} \\
        &\ |\ e_1.e_2        & \textrm{(projection)} \\
        &\ |\ e_1.e_2 = e_3  & \textrm{(assignment)} \\
        &\ |\ \texttt{if}\ e_1\ e_2\ e_3 & \textrm{(conditional)} \\
        &\ |\ e_1(e_2)       & \textrm{(application)} \\
        &\ |\ e_1; e_2       & \textrm{(sequencing)} \\
%        &\ |\ (e)          & \textrm{(grouping)} \\
        &\ |\ \texttt{function}\ x_{name}\ e_{type}\ (x_1, x_2, \cdots)\ e_{body} & \textrm{(method definition)}
\end{align*}
\end{singlespace}

The \texttt{new} construct resembles the \texttt{Dynamic} constructor
in past work on dynamic typing \cite{Abadi:1991:DTS:103135.103138}.
The main difference is that the tag is determined by evaluating an
expression.
%\footnote{Some agree that this qualifies as ``dynamic dependent typing''
%  (personal communication with Jean Yang, 2014). Others contend that this
%  terminology is not sensible.
%}
Why? The flippant answer is that we are already giving up decidable
type checking, so we might as well.
But what this really means is that constructing types is part of programming.
This can be used to request specific kinds of objects from an API, or
used as an implementation detail to tell the compiler which object
properties to track.
In practice, \texttt{new} is always wrapped in a constructor function, so
the programmer defining a data type is the only one who actually instantiates
it.
In fact, \texttt{new} for \emph{user-defined} data types is syntactically
available only within the code block that defines the type.
This provides some measure of data abstraction, since it is not possible
for user code to construct arbitrary instances.
However, tags themselves (the first argument to \texttt{new}) can be
constructed anywhere using nested applications of the syntax \texttt{T\{}$\cdots$\texttt{\}}.
Tags are a specific subset of data values whose tag is the built-in value
\texttt{Tag}.
Once so constructed, a tag can be passed to a constructor (which is just
a generic function) to obtain an instance if desired.

Constants are pre-built tagged values.

Types are a superset of tags that includes values generated by the
special tags \texttt{Abstract}, \texttt{Union}, and \texttt{UnionAll},
plus the special values \texttt{Any} and \texttt{Bottom}:

\vspace{-3ex}
\begin{singlespace}
\begin{align*}
  type\ ::=\ &\ \texttt{Bottom}\ |\ abstract\ |\ var \\
             &\ |\ \texttt{Union}\ type\ type \\
             &\ |\ \texttt{UnionAll}\ type\texttt{<:}var\texttt{<:}type\ type \\
             &\ |\ \texttt{Tag}\ x_{name}\ abstract_{super}\ value* \\
  abstract\ ::=\ &\ \texttt{Any}\ |\ \texttt{Abstract}\ x_{name}\ abstract_{super}\ value* \\
                 &\ |\ \texttt{Abstract}\ \texttt{Tuple}\ \texttt{Any}\ type*\ type\texttt{...}
\end{align*}
\end{singlespace}

\noindent
The last item is the special abstract varargs tuple type.

The language implicitly maps tags to data descriptions, and ensures that
the data part of a tagged value always conforms to the tag's description.
Mappings from tags to data descriptions are established by special type
declaration syntax. Data descriptions have the following grammar:

\vspace{-3ex}
\begin{singlespace}
\begin{align*}
data\ ::=\ &\ bit^n\ |\ ref\ |\ data*
\end{align*}
\end{singlespace}

\noindent
where $bit^n$ represents a string of $n$ bits, and $ref$ represents a reference
to a tagged data value. Data may be declared mutable, in which case its
representation is implicitly wrapped in a mutable cell. A built-in primitive
equality operator \texttt{===} is provided, based on $egal$ \cite{egal}
(mutable objects are compared by address, and immutable objects are compared
by directly comparing both the tag and data parts bit-for-bit, and recurring
through references to other immutable objects).

Functions are generally applied to more than one argument. In the application
syntax $e_1(e_2)$, $e_2$ is an implicitly-constructed tuple of all arguments.

We use the keyword \texttt{function} for method definitions for the sake of
familiarity, though \texttt{method} is arguably more appropriate. Method
definitions subsume lambda expressions. Each method definition modifies a
generic function named by the argument $x_{name}$. The function to extend is
specified by name rather than by value in order to make it easier to syntactically
restrict where functions can be extended. This, in turn, allows the language to
specify when new method definitions take effect, providing useful windows of
time within which methods do not change, allowing programs to be optimized more
effectively (and hopefully discouraging abusive and confusing run-time
method replacements).

% (TODO describe restrictions)

The equivalent of ordinary lambda expressions can be obtained by introducing
a unique local name and defining a single method on it.

Mutable lexical bindings are provided by the usual translation to operations
on mutable cells.

The signature, or specializer, of a method is obtained by evaluating $e_{type}$,
which must result in a type value as defined above. A method has $n$
formal argument names $x_i$. The specializer must be a subtype of the
varargs tuple type of length $n$. When a method is called, its formal argument
names are bound to corresponding elements of the argument tuple. If the
specializer is a vararg type, then the last argument name is bound to a
tuple of all trailing arguments.

%\subsection{A note on static typing}

% isomorphism between our types T and propositions ``term will be of type T''
% we will elide the difference
% trivially undecidable due to the definition of new()
% it is quite likely a useful static version could be developed. but
% we do not do that here, since our goals are (1) to develop the system
% for specialization & selection, not checking, and (2) to emphasize
% that no amount of ``dynamism'' need be given up.

% static type systems begin with errors we want to exclude, then design
% restrictions to make that possible, then go on to show that, indeed,
% most useful programs can still be written.

% one reason we skip static checking is to reverse this process:
% first see what kinds of type behavior technical users want in their
% programs, then identify and quantify any regularities later.


\section{Type system}

Our goal is to design a type system useful for describing method applicability,
and (similarly) for describing classes of values for which to specialize code.
Set-theoretic types are a natural basis for such a system.
A set-theoretic type is a symbolic expression that denotes a set of values.
In our case, these correspond to the sets of values methods are intended to apply
to, or the sets of values supported by compiler-generated method specializations.
Since set theory is widely understood, the use of such types tends to be intuitive.

These types
are less coupled to the languages they are used with, since one may design
a value domain and set relations within it without yet considering how types
relate to program terms \cite{1029823} \cite{Castagna:2005:GIS:1069774.1069793}.
Since our goals only include
performance and expressiveness, we simply skip the later steps for now, and do
not address how to type-check terms (or, indeed, the question of whether checking
is even possible).

To avoid the dual traps of ``wasted power'' and divergence, the system we use
must have a decidable subtype relation, and must be closed under data-flow operations
(meet, join, and widen). It must also lend itself to a reasonable definition of
specificity, so that methods can be ordered automatically (a necessary property for
extensibility). These requirements are fairly strict, but still admit many possible
designs. The one we present here is aimed at providing the minimum level of
sophistication needed to yield a language that feels ``powerful'' to most modern
programmers. Beginning with the simplest possible system, we added features as
needed either to satsify the aforementioned closure properties, or to allow us to
write method definitions that seemed particularly useful (as it turns out, these
two considerations lead to essentially the same features). The presentation that
follows will partially reproduce the order of this design process.

We will define our types by formally describing their denotations as sets.
We use the notation $\llbracket T \rrbracket$ for the set denotation of
type expression $T$.
Concrete language syntax and terminal symbols of the type expression grammar
are written in typewriter font, and metasymbols are written in mathematical italic.
First there is a universal type \texttt{Any}, an empty type \texttt{Bottom}, and
a partial order $\leq$:

\vspace{-2ex}
\begin{align*}
  \llbracket \texttt{Any} \rrbracket &= \mathcal{D} \\
  \llbracket \texttt{Bottom} \rrbracket &= \emptyset \\
  T \leq S &\Leftrightarrow \llbracket T \rrbracket \subseteq \llbracket S \rrbracket
\end{align*}

\noindent
where $\mathcal{D}$ represents the domain of all values.

Next we add data objects with structured tags.
The tag of a value is accessed with \texttt{typeof(x)}.
Each tag consists of a declared type name and some number of sub-expressions,
written as \texttt{Name\{}$E_1, \cdots, E_n$\texttt{\}}.
The center dots ($\cdots$) are meta-syntactic and represent a sequence of expressions.
Tag types may have declared supertypes (written as \texttt{super(T)}).
Any type used as a supertype must be declared as abstract, meaning it
cannot have direct instances.

\vspace{-2ex}
\begin{align*}
  \llbracket \texttt{Name\{}\cdots\texttt{\}} \rrbracket &= \{ x\mid \texttt{typeof(}x\texttt{)} = \texttt{Name\{}\cdots\texttt{\}} \} \\
  \llbracket \texttt{Abstract\{}\cdots\texttt{\}} \rrbracket &= \bigcup_{\texttt{super(}T\texttt{)} = \texttt{Abstract\{}\cdots\texttt{\}}} \llbracket T \rrbracket
\end{align*}

These types closely resemble the classes of an object-oriented language with
generic (parametric) types, invariant type parameters, and no concrete inheritance.
We prefer parametric \emph{invariance} partly for reasons that have been addressed in the
literature \cite{Day:1995:SVC:217838.217852}.
Invariance preserves the property that the only subtypes of a concrete type are \texttt{Bottom}
and itself.
This is important given how we map types to data representations: an \texttt{Array\{Int\}}
cannot also be an \texttt{Array\{Any\}}, since those types imply different
representations.
If we tried to use covariance despite this, there would have to be some \emph{other}
notion of which type a value \emph{really} had, which would be unsatisfyingly
complex
(tuples are a special case where covariance works, because each component type need
only refer to single value, so there is no need for concrete
tuple types with non-concrete parameters).
We also find that most uses of covariance are more flexibly
handled by union type connectives (introduced below).

Next we add conventional product (tuple) types, which are used to represent the
arguments to methods. These are almost identical to the nominal types defined above,
but are different in two ways: they are \emph{covariant} in their parameters, and permit
a special form ending in three dots (\texttt{...}) that denotes any number of trailing
elements:

\vspace{-2ex}
\begin{align*}
  \llbracket \texttt{Tuple\{}P_1,\cdots,P_n\texttt{\}} \rrbracket &= \prod_{1\leq i \leq n} \llbracket P_i \rrbracket \\
  \llbracket \texttt{Tuple\{}\cdots,P_n\texttt{...\}} \rrbracket, n\geq 1 &= \bigcup_{i\geq n-1} \llbracket \texttt{Tuple\{}\cdots,P_n^i\texttt{\}} \rrbracket
  %\llbracket \texttt{Tuple\{}\cdots\texttt{\}} \rrbracket \cup \llbracket \texttt{Tuple\{}\cdots,P_n\texttt{\}} \rrbracket \cup \llbracket \texttt{Tuple\{}\cdots,P_n,P_n\texttt{...\}} \rrbracket \\
\end{align*}

\noindent
$P_n^i$ represents $i$ repetitions of the final element $P_n$ of the type expression.

The abstract tuple types ending in \texttt{...} correspond to variadic methods, which
provide convenient interfaces for tasks like concatenating any number of arrays.
Multiple dispatch has been formulated as dispatch on tuple types before \cite{Leavens1998}.
This formulation has the advantage that \emph{any} type that is a subtype of a
tuple type can be used to express the signature of a method. It also makes the system
simpler and more reflective, since subtype queries can be used to ask questions about
methods.

The types introduced so far would be perfectly sufficient for many programs, and are
roughly equal in power to several multiple dispatch systems that have been designed
before. However, these types are not closed under data-flow operations. For example,
when the two branches of a conditional expression yield different types, a program
analysis must compute the union of those types to derive the type of the conditional.
The above types are not closed under set union. We therefore add the following
type connective:

\[
  \llbracket \texttt{Union\{}A,B\texttt{\}} \rrbracket = \llbracket A \rrbracket \cup \llbracket B \rrbracket \\
\]

As if by coincidence, \texttt{Union} types are also tremendously useful for expressing
method dispatch. For example, if a certain method applies to all 32-bit integers regardless
of whether they are signed or unsigned, it can be specialized for \texttt{Union\{Int32,UInt32\}}.

\texttt{Union} types are easy to understand, but complicate the type system considerably.
To see this, notice that they provide an unlimited number of ways to rewrite any type.
For example a type \texttt{T} can always be rewritten as \texttt{Union\{T,Bottom\}}, or
\texttt{Union\{Bottom,Union\{T,Bottom\}\}}, etc. Any code that processes types must
``understand'' these equivalences. Covariant constructors (tuples in our case)
also distribute over \texttt{Union} types, providing even more ways to rewrite types:

\[
\texttt{Tuple\{Union\{A,B\},C\}} = \texttt{Union\{Tuple\{A,C\},Tuple\{B,C\}\}}
\]

This is one of a few reasons that union types are often considered undesirable.
When used with type inference, such types can grow without bound, possibly leading
to slow or even non-terminating compilation. Their occurrence also typically
corresponds to cases that would fail most static type checkers. Yet from the
perspectives of both data-flow analysis and method specialization, they are
perfectly natural and even essential \cite{abstractinterp}
\cite{Igarashi} \cite{Smith:2008:JTI:1449764.1449804}.

The next problem we need to solve arises from data-flow analysis of
the \texttt{new} construct.
When a type constructor \texttt{C} is applied to a type
$S$ that is known only approximately at compile time, the type \texttt{C\{}$S$\texttt{\}}
does not correctly represent the result.
%if \texttt{C} is invariant.
The correct result would be the union of all types \texttt{C\{}$T$\texttt{\}}
where $T\leq S$.
Interestingly, there is again a corresponding need for such types in method
dispatch. Often one has, for example, a method that applies to arrays of any
kind of integer (\texttt{Array\{Int32\}}, \texttt{Array\{Int64\}}, etc.).
These cases can be expressed using a \texttt{UnionAll} connective, which denotes
an iterated union of a type expression for all values of a parameter in a specified
range:

\[
  \llbracket \texttt{UnionAll }L\texttt{<:T<:}U\ A \rrbracket = \bigcup_{L \leq T \leq U} \llbracket A[T/\texttt{T}] \rrbracket
\]

% TODO: The inclusion of lower bounds might make subtyping undecidable?

This is equivalent to an existential type \cite{boundedquant};
for each concrete subtype of it there exists a corresponding $T$.
Anecdotally, programmers often find existential types confusing.
We prefer the union interpretation because we are describing sets of values;
the notion of ``there exists'' can be semantically misleading since it sounds like
only a single $T$ value might be involved.

%Conjecture: these types are intuitive to dispatch on because they correspond
%to program behavior in the same way that dataflow analysis approximates program
%behavior.

% $T=S \longleftrightarrow (T\leq S) \wedge (S\leq T)$.

\subsection{Examples}

\texttt{UnionAll} types are quite expressive. In combination with nominal
types they can describe groups of containers such as
\texttt{UnionAll T<:Number Array\{Array\{T\}\}} (all arrays of arrays of
some kind of number) or
\texttt{Array\{UnionAll T<:Number Array\{T\}\}} (an array of arrays of
potentially different types of number).

In combination with tuple types, \texttt{UnionAll} types provide powerful
method dispatch specifications. For example
\texttt{UnionAll T Tuple\{Array\{T\},Int,T\}} matches three arguments:
an array, an integer, and a value that is an instance of the array's
element type. This is a natural signature for a method that assigns a
value to a given index within an array.


\subsection{Type constructors}

It is important for any proposed high-level technical computing language to be
simple and approachable, since otherwise the value over established
powerful-but-complex languages like C++ is less clear.
In particular, type parameters raise usability concerns.
Needing to write parameters along with every type is verbose, and requires users
to know more about the type system and to know more details of particular
types (how many parameters they have and what each one means).
Furthermore, in many contexts type parameters are not directly relevant.
For example, a large amount of code operates on \texttt{Array}s of any
element type, and in these cases it should be possible to ignore type parameters.

Consider \texttt{Array\{T\}}, the type of arrays with element type \texttt{T}.
In most languages with parametric types, the identifier \texttt{Array} would
refer to a type constructor, i.e. a type of a different \emph{kind} than
ordinary types like \texttt{Int} or \texttt{Array\{Int\}}.
Instead, we find it intuitive and appealing for \texttt{Array} to refer to
any kind of array, so that a declaration such as \texttt{x::Array} simply
asserts \texttt{x} to be some kind of array. In other words,

\vspace{-2ex}
\[
\texttt{Array} = \texttt{UnionAll T Array$^\prime$\{T\}}
\]

\noindent
where \texttt{Array$^\prime$} refers to a hidden, internal type constructor.
The \texttt{\{ \}} syntax can then be used to instantiate a \texttt{UnionAll}
type at a particular parameter value.

\subsection{Associated types and type computation}

Multiple dispatch systems have often only dispatched on the classes of
arguments.
This makes it appear necessary to introduce a separate
template system to handle other kinds of parameterization.
Our type system makes it easy to match the expressive features of
templates using dispatch.


\subsection{Subtyping}

\begin{figure}
  \begin{center}
    \def\arraystretch{3.5}
    \begin{tabular}{|c|}\hline
      \begin{tabular}{ccc}
        \AxiomC{$_A^B X^L, \Gamma\ \vdash\ T \leq S$}
        \UnaryInfC{$\Gamma\ \vdash\ \exists$ $_A^B X . T\ \leq\ S$}
        \DisplayProof

        \hspace{3ex}

        &

        \AxiomC{$_A^BX^R, \Gamma\ \vdash\ T \leq S$}
        \UnaryInfC{$\Gamma\ \vdash\ T\ \leq\ \exists$ $_A^B X . S$}
        \DisplayProof

        \hspace{3ex}

        &

        \AxiomC{}
        \UnaryInfC{$\Gamma\ \vdash\ X \leq X$}
        \DisplayProof
      \end{tabular}

      \\[8pt]

      \begin{tabular}{cc}
        \AxiomC{$^BX^L,\Gamma\ \vdash\ B \leq T$}
        \UnaryInfC{$^BX^L,\Gamma\ \vdash\ X \leq T$}
        \DisplayProof

        \hspace{3ex}

        &

        \AxiomC{$_AX^L,\Gamma\ \vdash\ T \leq A$}
        \UnaryInfC{$_AX^L,\Gamma\ \vdash\ T \leq X$}
        \DisplayProof

        \hspace{3ex}

        \\[8pt]

        \AxiomC{$^BX^L,{} _AY^L,\Gamma\ \vdash\ B \leq Y\ \vee\ X \leq A$}
        \UnaryInfC{$^BX^L,{} _AY^L,\Gamma\ \vdash\ X \leq Y$}
        \DisplayProof

        \hspace{4ex}

        &

        \AxiomC{$_A^BX^R,{} Y^R,\Gamma\ \vdash\ B \leq A$}
        \UnaryInfC{$_A^BX^R,{} Y^R,\Gamma\ \vdash\ X \leq Y$}
        \DisplayProof

        \\[8pt]

        \AxiomC{$_A^BX^R,\Gamma\ \vdash\ A \leq T$}
        \UnaryInfC{$_A^TX^R,\Gamma\ \vdash\ X \leq T$}
        \DisplayProof

        \hspace{4ex}

        &

        \AxiomC{$_A^BX^R,\Gamma\ \vdash\ T \leq B$}
        \UnaryInfC{$_{A \cup T}^{\ \ \ B}X^R,\Gamma\ \vdash\ T \leq X$}
        \DisplayProof

        \\[8pt]
      \end{tabular}
      \\
      \hline
    \end{tabular}
  \end{center}
  \caption{
    Subtyping algorithm for \texttt{UnionAll} ($\exists$) and variables.
    $X$ and $Y$ are variables, $A$, $B$, $T$, and $S$ are types.
    $_A^BX$ means $X$ has lower bound $A$ and upper bound $B$.
    %$^R$ and $^L$ are used to track whether a variable originated on
    %the right or on the left.
  }
  \label{subtvars}
\end{figure}


Deciding subtyping for base types is straightforward: \texttt{Bottom} is
a subtype of everything, everything is a subtype of \texttt{Any}, and
tuple types are compared component-wise. The invariant parameters
of tag types are compared in both directions: to check
$\texttt{A\{B\}}\leq \texttt{A\{C\}}$, check $\texttt{B}\leq\texttt{C}$ and
then $\texttt{C}\leq\texttt{B}$. In fact, the algorithm depends on these
checks being done in this order, as we will see in a moment.

Checking union types is a bit harder. When a union $A\cup B$ occurs in the
algorithm, we need to nondeterministically replace it with either $A$ or
$B$. The rule is that for all such choices on the left of $\leq$, there
must exist a set of choices on the right such that the rest of the
algorithm answers ``yes''. This can be implemented by keeping a stack of
decision points, and looping over all possibilities with an outer
for-all loop and an inner there-exists loop. We speak of ``decision points''
instead of individual unions, since in a type like \texttt{Tuple\{Union\{A,B\}...\}}
a single union might be compared many times.

The algorithm for \texttt{UnionAll} and variables is shown in figure~\ref{subtvars}.
The first row says to handle a \texttt{UnionAll} by extending the environment
with its variable, marked according to which side of $\leq$ it came from,
and then recurring into the body.
In analogy to union types, we need to check that for all variable values on
the left, there exists a value on the right such that the relation holds.
The for-all side is relatively easy to implement, since we can just use
a variable's bounds as proxies for its possible values (this is shown in
the second row of the figure).
We implement the there-exists side by narrowing a variable's bounds
(raising the lower bound and lowering the upper bound). The relation holds
as long as the bounds remain consistent (i.e. lower bound $\leq$
upper bound). The algorithm assumes that all input types are well-formed,
which includes variable lower bounds always being less than or equal to
upper bounds.

However, at this point (starting in the third row, second column) the
algorithm appears asymmetric. This is a result of exploiting the lack of
contravariant constructors. No contravariance means that every time a
right-side variable appears on the \emph{left} side of a comparison,
it must be because it occurs in invariant position, and the steps outlined
in the first paragraph of this section have ``flipped'' the order
(comparing both $\texttt{B}\leq\texttt{C}$ and $\texttt{C}\leq\texttt{B}$).
This explains the odd rule for comparing two right-side variables:
this case can only occur with differently-nested \texttt{UnionAll}s and
invariant constructors, in which case the relation holds only if
all involved bounds are equal.
By symmetry, one would expect the bottom-left rule in the figure to
update $X$'s upper bound to $B\cap T$. But because of invariance,
$T\leq B$ has already been checked by the bottom-right rule in the
figure. Therefore $B\cap T = T$. This is the reason the ``forward''
direction of the comparison needs to be checked first: otherwise,
we would have updated $B$ to equal $T$ already and the $T\leq B$
comparison would become vacuous. Alternatively, we could actually
compute $B\cap T$. However there is reason to suspect that serious
trouble lies that way. We would need either to add intersection types
to the system, or compute a meet without them. Either way, the
algorithm would become much more complex and, judging by past
results, very likely undecidable.

TODO: termination and correctness.

These subtyping rules are very likely $\Pi_2^{\textrm{P}}$-hard.
Checking a subtype relation with unions requires checking that for all choices
on the left, there exists a choice on the right that makes the relation hold.
This matches the quantifier structure of 2-TQBF problems of the form
$\forall x_i . \exists y_i . F$ where $F$ is a boolean formula. If the formula
is rewritten in conjunctive normal form, it corresponds to subtype checking
between two tuple types, where the relation must hold for each pair of
corresponding types. Now use a type \texttt{N\{}$x$\texttt{\}} to
represent $\neg x$. The clause $(x_i \vee y_i)$ can be translated to
$x_i\ \texttt{<:\ Union\{N\{}y_i\texttt{\}, True\}}$ (where the $x_i$ and
$y_i$ are type variables bound by \texttt{UnionAll} on the left and right,
respectively).
We have not worked out the details, but this sketch is reasonably
convincing. $\Pi_2^{\textrm{P}}$ is only the most obvious
reduction to try; it is possible our system equals PSPACE or even greater,
as has often been the case for subtyping systems like ours.


\subsection{Type system variants}

Our design criteria do not identify a unique type system.
Many variants are possible.
The following features would probably be fairly straightforward to add:

\vspace{-3ex}
\begin{singlespace}
\begin{itemize}
\item Structurally-subtyped records
\item $\mu$-recursive types (regular trees)
\item Regular types (allowing \texttt{...} in more places)
\end{itemize}
\end{singlespace}

\noindent
The following features would be difficult to add, or possibly break decidability
of subtyping:

\vspace{-3ex}
\begin{singlespace}
\begin{itemize}
\item Arrow types
\item Negations
\item Intersections, multiple inheritance
\item Universal quantifiers
%\item arbitrary predicates, theory of natural numbers, etc.
\end{itemize}
\end{singlespace}


\section{Dispatch mechanism}

\subsection{Ambiguities}

\subsection{Constructors}


\section{Higher-order programming}

Generic functions are first-class objects, and so can be passed as arguments
just as in any dynamically-typed language with first-class functions.
However, assigning useful type tags to generic functions and deciding how
they should dispatch is not so simple. Past work has often described the
types of generic functions using the ``intersection of arrows'' formalism
\cite{RonchiDellaRocca:1988:PTS:55079.55086} \cite{Dunfield:2012:EIU:2364527.2364534}
\cite{boundedquant} \cite{Castagna:1995:COF:203496.203510}. Since an ordinary
function has an arrow type $A\rightarrow B$ describing how it maps arguments
$A$ to results $B$, a function with multiple definitions can natually be
considered to have multiple such types. For example, a \texttt{sin} function
with the following two definitions:

\begin{singlespace}
\begin{lstlisting}[style=customjulia]
sin(x::Float64) = # compute sine of x in double precision
sin(v::Vector) = map(sin, v)
\end{lstlisting}
\end{singlespace}

\noindent
could have the type $(\texttt{Float64}\rightarrow\texttt{Float64})\cap(\texttt{Vector}\rightarrow\texttt{Vector})$. The intuition is that this \texttt{sin} function can be
used both where a $\texttt{Float64}\rightarrow\texttt{Float64}$ function
is expected and where a $\texttt{Vector}\rightarrow\texttt{Vector}$ function is expected,
and therefore its type is the intersection of these types.

This approach is effective for statically checking uses of generic
functions: anywhere a function goes, we must keep track of which arrow
types it ``contains'' in order to be sure that at least one matches
every call site and allows the surrounding code to type check.
However, despite the naturalness of this typing of generic functions,
this formulation is quite problematic for dispatch and code specialization
(not to mention that it might make subtyping undecidable).

\subsection{Problems for code selection}

Consider what happens when we try to define an integration function:

\begin{singlespace}
\begin{lstlisting}[style=customjulia]
# 1-d integration of a real-valued function
integrate(f::Float64->Float64, x0, x1)

# multi-dimensional integration of a vector-valued function
integrate(f::Vector->Vector, v0, v1)
\end{lstlisting}
\end{singlespace}

\noindent
The \texttt{->} is not real Julia syntax, but is assumed for the sake of
this example.
Here we wish to select a different integration routine based on what
kind of function is to be integrated.
However, these definitions are ambiguous with respect to the \texttt{sin}
function defined above.
Of course, the potential for method ambiguities existed already.
However this sort of ambiguity is introduced \emph{non-locally} ---
it cannot be detected when the \texttt{integrate} methods are defined.

%re: selection:
%- feels like the wrong abstraction. GFs are the means of selecting behavior,
%- changes over time
%- might depend on inference results

Such a non-local introduction of ambiguity is a special case of the
general problem that a generic function's type would change depending
on what definitions have been added, which depends e.g. on which libraries
have been loaded. This does not feel like the right abstraction:
type tags are supposed to form a ``ground truth'' about objects against
which program behavior can be selected. Though generic functions change
with the addition of methods, it would be more satisfying for their
types to somehow reflect an intrinsic, unchanging property.

An additional minor problem with the intersection of arrows
interpretation is that we have found, in practice, that Julia methods
often have a large number of definitions. For example, the \texttt{+}
function in Julia v0.3.4 has 117 definitions, and in a more recent
development version with more functionality, it has 150 methods. An
intersection of 150 types would be unwieldy, even if only
used when debugging the compiler.

A slightly different approach we might try would be to immitate
the types of higher-order functions in traditional
statically-typed functional languages. For example, we might wish
to write \texttt{map} as follows:

\begin{singlespace}
\begin{lstlisting}[style=customjulia]
map{A,B}(f::A->B, x::List{A}) =
  isempty(x) ? List{B}() : List{B}(f(head(x)), map(f, tail(x)))
\end{lstlisting}
\end{singlespace}

The idea is for the first argument to match any function, and not use
the arrow type for dispatch, thereby avoiding ambiguity problems.
Instead, immediately after method selection, values for \texttt{A} and
\texttt{B} would be determined using the element type of \texttt{x}
and the table of definitions of \texttt{f}.

Unfortunately it is not clear how exactly \texttt{B} should be
determined. We could require return type declarations on every method,
but this would adversely affect usability (such declarations would also
be helpful if we wanted to dispatch on arrow types, though they would
not solve the ambiguity problem). Or we could use type inference
of \texttt{f} on argument type \texttt{A}. This would not work very
well, since the result would depend on partly-arbitrary heuristics.
Such heuristics are fine for analyzing a program, but
are not appropriate for determining the value of a user-visible
variable, as this would make program behavior unpredictable.

\subsection{Problems for code specialization}

For code specialization to be effective, it must eliminate as many
irrelevant cases as possible. Intersection types seem to be naturally
opposed to this process, since they have the ability to
generate infinite descending chains of ever-more-specific function
types by tacking on more terms with $\cap$. There would be no such
thing as a maximally-specific function type. In particular, it would be
hard to express that a function has exactly one definition,
which is an especially important case for optimizing code.

For example, say we have a definition \texttt{f(g::String->String)},
and a function \texttt{h} with a single $\texttt{Int}\rightarrow\texttt{Int}$ definition.
Naturally, \texttt{f} is not applicable to \texttt{h}.
However, given the call site \texttt{f(h)}, we are forced to conclude
that \texttt{f} might be called with a function of type
\mbox{$(\texttt{Int}\rightarrow\texttt{Int})\cap(\texttt{String}\rightarrow\texttt{String})$},
since in general $\texttt{Int}\rightarrow\texttt{Int}$ might be only an
approximation of the true type of the argument.
% TODO so? wouldn't be able to exclude the method

%re: specialization:
%specialization wants to exclude irrelevant cases, and intersections
%*manufacture* irrelevant cases!
%- GFs have many methods and no two are the same in practice
%- there is no way to restrict a function to one definition

The other major concern when specializing code is whether, having generated code
for a certain type, we would be able to reuse that code often enough for the
effort to be worthwhile.
In the case of arrow types, this equates to asking how often generic functions
share the same set of signatures.
This question can be answered empirically.
Studying the Julia \texttt{Base} library as of this writing, there are 1059
generic functions. We examined all 560211 pairs of functions; summary
statistics are shown in figure~\ref{fig:matchingfuncs}.
Overall, it is rare for functions to share type signatures.
Many of the 85 functions with matches (meaning there exists some other function
with the same type) are predicates, which all have types similar to
$\texttt{Any}\rightarrow \texttt{Bool}$. The mean of 0.23 means that if we
pick a function uniformly at random, on average 0.23 other functions will
match it.
The return types compared here depend on our heuristic type inference algorithm,
so it useful to exclude them in order to get an upper bound.
If we do that, and only consider arguments, the mean number of matches rises to 1.73.

\begin{figure}
  \begin{center}
    \begin{tabular}{|l|r|r|r|}\hline
    &  \textbf{matching pairs} & \textbf{GFs with matches} & \textbf{mean} \\
      \hline \hline
arguments only             & 1831 (0.327\%)  &   329       &          1.73 \\
\hline
arguments and return types &  241 (0.043\%)  &   85        &          0.23 \\
\hline
\end{tabular}
\end{center}
  \caption{
    Number and percentage of pairs of functions with matching arguments, or
    matching arguments and return types. The second column gives the number of
    functions that have matches. The third column gives the
    mean number of matches per function.
  }
  \label{fig:matchingfuncs}
\end{figure}

The specific example of the \texttt{sin} and \texttt{cos} functions provides
some intuition for why there are so few matches.
One would guess that the type behavior of these functions would be identical,
however the above evaluation showed this not to be the case.
The reason is that the functions have definitions to make them operate
elementwise on both dense and sparse arrays.
\texttt{sin} maps zero to zero, but \texttt{cos} maps zero to one,
so \texttt{sin} of a sparse array gives a sparse array, but
\texttt{cos} of a sparse array gives a dense array.
This is indicative of the general ``messiness'' of convenient real-world
libraries for technical computing.
% todo conclusion


\subsection{Possible solutions}

%identity-typing
The general lack of sharing of generic function types suggests the first
possible solution: give each generic function a new type that is uniquely
associated with it. For example, the type of \texttt{sin} would be
\texttt{GenericFunction\{sin\}}. This type merely identifies the function
in question, and says nothing more about its behavior. It is easy to read,
and easily specific enough to avoid ambiguity and specialization
problems. It does \emph{not} immediately solve the above problem of
determining the result type of \texttt{map}. However there are
corresponding performance benefits, since specializing code for a
specific function argument naturally lends itself to inlining.

% todo: note that this is just a special case of specializing on a
% constant. however this analysis argues for specializing on constant
% functions by default, where specializing on other values by default
% is less surely a good idea.

%nominal function types
Another approach that is especially relevant to technical computing is to use
nominal function types.
In mathematics, the argument and return types of a function are often among its
least interesting properties.
In some domains, for example, all functions can implicitly be assumed
$\mathbb{R}\rightarrow\mathbb{R}$, and the interesting property might be
what order of integrable singularity is present (see section~\ref{sec:BEM} for
an application), or what dimension of linear operator the function represents.
The idea of nominal function types is to describe the properties of interest
using a data object, and then allow that data object to be treated as a
function, i.e. ``called''. Some object-oriented languages call such an object
a \emph{functor}.

Julia accomodates this approach with a small adjustment to the evaluation
rules: in the application syntax $e_1(e_2)$ when $e_1$ is not a function,
evaluate $\texttt{call}(e_1,e_2)$ instead, where \texttt{call} is
a particular generic function known to the system.
Now we can define

\begin{singlespace}
\begin{lstlisting}[style=customjulia]

\end{lstlisting}
\end{singlespace}


\begin{singlespace}
\begin{lstlisting}[style=customjulia]
immutable Arrow{A,B}
    f
end

call{A,B}(a::Arrow{A,B}, x::A) = a.f(x)::B
\end{lstlisting}
\end{singlespace}

% nominal types that classify algorithms, e.g. whether a sort is stable


% arrow types without intersections, used to ``slice'' generic functions

% (generic functions with declared overall types)

\subsection{Implementing \texttt{map}}



\section{Performance model}

\subsection{Type inference}

\subsection{Specialization}

% ? \section{Staged programming}
