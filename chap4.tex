\chapter{The Julia Approach}

This chapter will illustrate how we implement key features of technical computing
systems using our methodology.


\section{Numbers}

Despite their apparent simplicity, numbers in fact tend to be among the most
complex features of a language. Numeric types usually need to be a special
case: in a typical language with built-in numeric types, describing their
behavior is beyond the expressive power of the language itself. For example,
in C arithmetic operators like \texttt{+} accept multiple types of arguments
(ints and floats), but no user-defined C function can do this (this situation
is of course improved in C++). In Python, a special arrangement is made for
\texttt{+} to call either an \texttt{\_\_add\_\_} or \texttt{\_\_radd\_\_} method, effectively
providing double-dispatch for arithmetic in a language that is idiomatically
single-dispatch.



\section{Arrays}



\section{Units}

