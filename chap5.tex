\chapter{Implementation}


\section{Classification of languages}

It is helpful to begin with a rather coarse classification of programming
languages, according to how expressive their type systems are, and whether
their type systems are dynamic or static:
\\
\begin{tabular}{|c||c|c|}
\hline 
 & More types & Fewer types\tabularnewline
\hline 
\hline 
Dynamic & Dylan, Julia & Scheme, Python, MATLAB\tabularnewline
\hline 
Static & ML, Haskell, Scala & C\tabularnewline
\hline 
\end{tabular}
\\

The lower right corner tend to contain older languages. The upper right corner
contains many popular ``dynamic'' languages. The lower left corner contains
many modern languages resulting from research on static type systems.
We are most interested in the upper left corner, which is notable for being
rather sparsely populated. It has been generally believed that dynamic
languages do not ``need'' types, or that there is no point in talking about
types if they are not going to be checked at compile time. These views have
some merit, but as a result the top-left corner of this design space has
been seriously under-explored.


\section{Multiple dispatch}

CHART of different multiple dispatches. classify them

Why ``just overloading'' is not enough. You intercept every operation and
rewrite it, kind of an escape hatch for a language you don't like. If
somebody else also tries to do this, there won't necessarily be any
coherence.


A helpful way to classify languages with some kind of generic programming
support is to look at which language constructs are generic. For example,
in C++ the syntax \texttt{object.method(x)} is generic: a programmer can
get it to do different things by supplying different values for
\texttt{object}. The C++ syntax \texttt{f(a,b)} or \texttt{a+b} is
usually not generic, but can be overloaded by supplying definitions for
different argument types. However, this overloading only uses compile-time
types (no run-time information), and so is essentially a form of renaming ---
it can be implemented by renaming each definition with a unique name, and
replacing overloaded calls with an appropriate name based on compile-time
types. This is a much weaker form of generic programming, since different
behaviors cannot be obtained by passing different values at run time.

For this reason, C++ programs are not fully generic: it is difficult to
substitute new behaviors for every part of a program. Some languages
take generic programming much further. For example, in MATLAB, \emph{every}
function is effectively generic, and can be overloaded by new classes.
This enables a programmer to ``intercept'' every function call (and therefore,
essentially everything a program does)



\section{Domain Theory}

In the 1960s Dana Scott asked how to assign meanings to programs,
which otherwise just appear to be lists of symbols.
For example, given a program computing the factorial function, we
want a process by which we can assign the meaning ``factorial'' to it.

This is about modeling the behavior of a program without running it.



The idea of analyzing computer programs began in earnest in the 1960s with
Dana Scott's invention of domain theory. A ``domain'' in this theory is a
partial order of sets of values that a program might manipulate. 
Domain theory models computation as follows: a program starts with no
information, the lowest point in the partial order (``bottom'').
Computation steps accumulate information, gradually moving higher through
the order. The advantage of this model, in essence, is that it provides a
way to think about the meaning of a program without running the program.
Even the ``result'' of a non-terminating program has a representation ---
the bottom element. Other elements of the partial order might refer to
intermediate results.


Domain theory gave rise to the study of denotational semantics and the
design of type systems. However, the original theory is quite general
and invites us to invent any domains we wish for any sort of language.
For example, given a program that outputs an integer, we might decide
that we only care whether this integer is even or odd. Then our posets
are the even and odd integers, and we will classify operations in the
program according to whether they are evenness-preserving and so on.
It should be clear that this sort of analysis, while clearly related
to type systems, is fairly different from what most programmers
think of as type checking.




\section{Dynamic type inference}

This is exciting because the type system can do useful computational work for you,
instead of only checking the work done by the rest of the program.

Joins of quantified types

``Diagonal dispatch''

(T,T) (Int, Real)

\textbackslash{}(Int,Int)/

Avoid hard coding logic into the compiler.


Exposing to the user what is part of the language anyway and not just
hiding it in the compiler.

Data flow analysis has to actually work

Taking things out of the language that break dataflow analysis

Local reasoning

You are imposing a type system anyway, even implicitly perhaps, so
might as well make that from the get-go.


\section{Objections to dynamic typing and counterarguments}


it may be that the ``power'' of a language is equal to the complexity
of the criteria used by the language's run-time dispatch mechanisms.

pre-OO: pointer indirection
OO: single dispatch, class hierarchies

Haskell: without typeclasses, a beautiful language, but one that nobody
would use.

I claim that compile-time abstractions do not count. The problem is that
at some point you have to *actually run the program*. Specifying the
behavior of a program is hard; this is where we need help from the
language.

