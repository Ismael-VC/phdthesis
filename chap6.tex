\chapter{Case studies}


\section{Numerical linear algebra}

Multiple dispatch on special matrices

29 LAPACK types via composition of 9 types, issue 8240


\section{Boundary-element method}

There are lots of general packages for FEM problems, but it is much more
difficult to create such a package for BEM problems. The method requires
integrating functions with singularities, many times in the inner loop
of code that builds the problem matrix. Integrating such functions
numerically on each iteration is much too slow. As a result, many
special-purpose implementations have been written by hand for different
problems.

Some recent work (TODO cite Homer) managed a more general solution,
using Mathematica to generate C++ code for different cases. This worked
well, but was difficult to implement and the resulting system is difficult
to use. We see the familiar pattern of using multiple languages and
code-generation techniques, with coordination of the overall process done
either manually or with ad-hoc scripts.

%%%%%%%%%%%%%%%%%%%%%%%%%%%%%%%%%%%%%%%%%%%%%%%%%%%%%%%%%%%%%%%%%%%%%%%%%%%

% writeup by SGJ

\section{Galerkin matrix assembly for singular kernels}

A typical problem in computational science is to form a discrete approximation of some infinite-dimensional linear operator $\mathcal{L}$ with some finite set of basis functions $\{ b_m \}$ via a Galerkin approach [refs], which leads to a matrix $L$ with entries $L_{mn} = \langle b_m, \mathcal{L} b_n \rangle = \langle b_m, b_n \rangle_\mathcal{L}$ where $\langle \cdot, \cdot \rangle$ denotes some inner product (e.g. $\langle u, v \rangle = \int u v$ is typical) and  $\langle \cdot, \cdot \rangle_\mathcal{L}$ is the \emph{bilinear form} of the problem.  Computing these matrix elements is known as the matrix \emph{assembly} step, and its performance is a crucial concern for solving partial differential equations (PDEs) and integral equations (IEs).

\subsection{The ``easy'' case: nonsingular assembly}

For example, in the finite-element method (FEM) [refs], the basis functions $b_m$ are typically low-order polynomials defined piecewise over geometric elements (typically triangles or tetrahedra), and $\mathcal{L}$ is typically a differential operator like $-\nabla \cdot c(x) \nabla$ for some coefficients $c(x)$, which leads to a bilinear form  $\langle b_m, b_n \rangle_\mathcal{L} = \int \nabla b_m \cdot c(x) \nabla b_n$ (after integration by parts).   Because the basis functions are localized and $\mathcal{L}$ consists of local operations, the matrix $L$ is sparse and $L_{mn}$ need only be computed for $m$ and $n$ corresponding to neighboring elements. Moreover, these integrals are straightforward to evaluate by standard cubature schemes because the integrands are \emph{nonsingular}: they typically have no divergences or discontinuities. In particular, because the functions $b_m$ and $c$ are usually smooth within a single element, one can use a fixed low-order cubature rule: you evaluate the integrand at a handful of precomputed points within each element, multiply by precomputed weights, and sum to obtain the approximate integral.

Even so, the basis functions and the coefficient function $c(x)$ may need to be evaluated tens of millions of times for even a moderate-size mesh in three dimensions, so production FEM implementations in traditional high-level dynamic languages such as Matlab and Python are forced to offload matrix assembly to external C and C++ code.  For example, the popular FEniCS [ref] and Firedrake [ref] FEM packages for Python both implement domain-specific compilers: a symbolic expression for the bilinear form is combined with fragments of user-specified C++ code to define functions like $c(x)$, compiled to C++ code, and then compiled to object code which is dynamically loaded. In Julia, we believe this could be simplified considerably because functions like $c(x)$ could be defined directly in Julia and code generation/compilation could be performed entirely within Julia without a C++ intermediary.  Indeed, preliminary experiments with pure Julia FEM implementations [ref http://www.codeproject.com/Articles/579983/Finite-Element-programming-in-Julia] have demonstrated performance comparable to sophisticated solutions like FEniCS in Python and FreeFem++ in C++ [ref]

\subsection{Singular assembly for integral operators}

A much more challenging case of Galerkin matrix assembly arises for
singular \emph{integral} operators $\mathcal{L}$, which act by
convolving their operand against a singular ``kernel'' function
$K(x)$: $u = \mathcal{L} v$ means that $u(x) = \int K(x - x') v(x')
dx'$.  For example, in electrostatics and other Poisson problems, the
kernel is $K(x) = 1/|x|$ in three dimensions and $\ln |x|$ in two
dimensions, while in scalar Helmholtz (wave) problems it is
$e^{ik|x|}/|x|$ in three dimensions and a Hankel function
$H^{(1)}_0(k|x|)$ in two dimensions.  Formally, Galerkin
discretizations lead to matrix assembly problems similar to those
above: $L_{mn} =: \langle b_m, \mathcal{L} b_n \rangle = \int b_m(x)
K(x - x') b_n(x') dx\,dx'$.  However, there are several important
differences from FEM:

\begin{itemize}

\item The kernel $K(x)$ nearly always diverges for $|x|=0$, which means that generic cubature schemes are either unacceptably inaccurate (for low-order schemes) or unacceptably costly (for adaptive high-order schemes, which require huge numbers of cubature points around the singularity), or both.

\item Integral operators typically arise for \emph{surface} integral
  equations (SIEs) [ref], and involve unknowns on a surface.  The
  analogue of the FEM discretization is then a boundary element method
  (BEM) [ref], which discretizes a surface into elements
  (e.g. triangles), with basis functions that are low-order
  polynomials defiend piecewise in the elements.  However, there are
  also volume integral equations (VIEs) which have FEM-like volumetric
  meshes and basis functions.

\item The matrix $L$ is typically dense, since $K$ is long-range.  For
  large problems, $L$ is often stored and applied implicitly via
  fast-multipole methods [refs] and similar schemes, but even in this
  case the diagonal $L_{mm}$ and the entries $L_{mn}$ for adjacent
  elements must typically be computed explicitly.  (Moreover, these
  are the integrals in which the $K$ singularity is present.)

\end{itemize}

These difficulties are part of the reason why there is currently \emph{no}
truly ``generic'' BEM software, analogous to FEniCS for FEM: essentially
all practical BEM code is written for a specific integral kernel and
a specific class of basis functions arising in a particular physical problem.
Changing anything about the kernel or the basis---for example, going
from two- to three-dimensional problems---is a major undertaking.

We believe that Julia should be an ideal platform on which to attack this
problem:

\begin{itemize}

\item Multiple dispatch allows the cubature scheme to be selected at compile-time based on the dimensionality, the degree of the singularity, the degree of the polynomial basis, and so on, and allows specialized schemes to be added easily for particular problems with no runtime penalty.

\item Staged functions allow computer-algebra systems to be invoked at
  compile-time to generate specialized cubature schemes for particular
  kernels.  New developments in BEM integration schemes [ref Homer] have
  provided efficient cubature-generation algorithms of this sort, but it
  has not yet been practical to integrate them with runtime code in
  a completely automated way.

\end{itemize}

A prototype implementation of this approach follows.


%%%%%%%%%%%%%%%%%%%%%%%%%%%%%%%%%%%%%%%%%%%%%%%%%%%%%%%%%%%%%%%%%%%%%%%%%%%

\section{Dates}

compare to python DateTime, compare code length


\section{JUMP}

Metaprogramming tools reused for symbolic algebra


\section{Computational geometry}

robust predicates (dispatch over points, lines) and VoronoiDelaunay.jl
benchmarked against CGAL


%\section{ACAS}

%\section{Parallel computing, e.g. parallel prefix}

%\section{IJulia}
