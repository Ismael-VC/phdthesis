\documentclass[12pt]{article}
\usepackage{fullpage}
\pagestyle{plain}

%%%%%%%%%%%%%%%%%%%%%%%%%%%%%%%%%%%%%%%%%%%%%%%%%%%%%%%%%%%%%%%%%%%%%%%%%%%%%%%
% Fill in the following information and compile to create your very
% own EECS thesis proposal.  Thesis not included.
%%%%%%%%%%%%%%%%%%%%%%%%%%%%%%%%%%%%%%%%%%%%%%%%%%%%%%%%%%%%%%%%%%%%%%%%%%%%%%%

\def\author{Jeffrey Werner Bezanson}
\def\addressone{25 Walbridge St. \#15}
\def\addresstwo{Allston, MA 02134}

\def\supervisor{Alan Edelman}

\def\title{Abstraction in Technical Computing}
\def\expectedcompletetiondate{January 9, 2015}
\def\briefproblemstatement{

Array-based programming environments are popular for scientific and
technical computing.
These systems consist of built-in function libraries paired with high-level
languages for interaction.
Although the libraries perform well, it is widely believed that scripting in these
languages is necessarily slow, and that only heroic feats of engineering can at
best partially ameliorate this problem.

In this thesis I argue that what is really needed is a more coherent
structure for this functionality. For this role I propose an abstraction
based on an extended version of generic functions. The novelty of this
mechanism is that it is both flexible enough to describe the wide
variety of behaviors users need in practice, while also providing
enough information to a compiler to yield good performance.


% Technical computing is characterized by great diversity and
% a rapid pace of change.

% dynamic flexibility is a good way to think about technical computing
% as a whole

% we don't know what people will want in the future, but
% a good way to try to provide for it is to at least be able
% to describe and implement what we want now. most systems
% can't even do this.

%Due to the great diversity and pace of change in technical applications,
%users would like to be able to extend and modify such systems.
%However, most of the functionality of these systems is ``built in'',
%% (rigidly defined?)
%and difficult to extend in a way that retains good performance.


%- adding a coherent structure to all the functionality, not just
%  an interface on top
%- develop a new methodology for creating one of these environments,
%  which extends to using it.


% must say something about: the key challenge is to provide the full
% rich behaviors that users expect.


%adding a coherent structure
%expressing everything with a uniform multimethod paradigm
%imagining multimethods as a solution to this, it also has to be
%extended
%if we extended multimethods in certain ways, they could solve this

%in order to motivate it you have to talk about technical computing,
%since that use case demands it.

%we want to implement technical computing within a uniform abstract
%framework. we invented a mechanism for doing that. 

%rather than taking multimethods as a fancy feature, we use it as the
%structural foundation of the system.

%expressive method signatures are simultaneously useful for the behavior
%they specify, and for the information they provide the compiler.

}


%%%%%%%%%%%%%%%%%%%%%%%%%%%%%%%%%%%%%%%%%%%%%%%%%%%%%%%%%%%%%%%%%%%%%%%%%%%%%%%
% Macros and whatnot to generate all the forms. 
% Macros written by David Liben-Nowell; last modified 7 February 2005.
%%%%%%%%%%%%%%%%%%%%%%%%%%%%%%%%%%%%%%%%%%%%%%%%%%%%%%%%%%%%%%%%%%%%%%%%%%%%%%%

\setlength{\parindent}{0pt}
\setlength{\parskip}{1em}
\def\signatureline{\rule{2.5in}{0.25pt}}

\def\mitheader#1{
\begin{center}
  \textsc{massachusetts institute of technology}\\
  Department of Electrical Engineering and Computer Science\\
  Cambridge, Massachusetts~~02139\\[2ex]
  
  \textbf{#1}
\end{center}
}

\def\tofromheader#1{

\begin{tabular}{ll}
To:  & Department Graduate Committee \\
From: & #1 \\
\end{tabular}

}



\def\thesisproposalform{
\begin{center}
Massachusetts Institute of Technology

Department of Electrical Engineering and Computer Science

\bigskip

Proposal for Thesis Research in Partial Fulfillment

of the Requirements for the Degree of 

Doctor of Philosophy
\end{center}

\bigskip

Title:  \title

Submitted by: \hfill
\begin{tabular}[t]{p{0.3\textwidth}p{0.4\textwidth}}
\author & \signatureline \\
\addressone & (Signature of author) \\
\addresstwo \\
\end{tabular}

Date of submission:  \today

Expected Date of Completion:  \expectedcompletetiondate

Laboratory where thesis will be done: CSAIL

Brief Statement of the Problem:

\briefproblemstatement 

 \newpage }

\begin{document}

\thesisproposalform

\newpage

\section{Introduction}

Technical computing is an especially rich application area for programming
languages to address. The observation that a language ``is likely to be used to
solve problems which its designer did not envision'' \cite{Liskov:1974pb}
applies doubly: the programmers are domain experts, and are likely to be
doing original research. This kind of programming is also data- and
compute-intensive, leading to conflicting requirements of performance and
flexibility. This thesis will discuss the reasons behind this conflict and
the existing solutions, and propose a new solution.

The leading approach to the performance-flexibility compromise today is
essentially polyglot programming: using multiple languages together to
combine their strengths. Due to the unusual requirements of technical
computing, it is surprisingly common for one of the languages to be
designed by domain experts who concluded that no general-purpose
languages were suitable for their problems. Examples include R \cite{Rlang}
for statistics, MATLAB\textregistered, Octave \cite{octave},
and SciLab \cite{scilab} for engineering and linear algebra, APL \cite{APL}
and ZPL \cite{ZPL} for array operations, Torch7 \cite{collobert2011torch7}
for machine learning, Lush \cite{bottou2002lush} for
signal and image processing, and quite a few others. This diversity is
not necessarily a bad thing, but it is telling us something. While we
should not pretend to be able to capture all the features and advantages
of these systems in a single new system, patterns do emerge in the
general characteristics designers want from these systems. This thesis
will attempt to isolate these characteristics and address them
with a new design.

It is possible to take the view that a new language is not really needed.
Another promising approach, for example, is to deploy
techniques to improve the performance of existing popular high-level
languages. Many of these techniques require
no source modification: static analysis, JIT compilation, tracing \cite{tracingjit},
or moving more functionality into low-level libraries. Some techniques involve
slight source modifications: optional type annotations, decorators that
invoke code transformations, or alternate ``dialects'' such as Cython
\cite{Behnel:2011}.

These techniques are of great value. For most applications they are a boon.
In the case of JavaScript, for example, they have transformed the web
dramatically. However, technical computing is a bit different, and for
this area significant problems remain in both performance and flexibility.
JIT compilers tend to suffer from a ``performance ceiling'': speculative
optimizations require run-time checks, making it difficult to fully match
the performance of statically-compiled languages (C, C++, Fortran).
At the same time, technical libraries often embody highly complex behaviors
that a compiler nonetheless needs to understand to obtain good
performance. In many cases it may not be realistic to infer these behaviors
from imperative code.

Our solution is to develop a language and library-writing style based on
heavy use of multiple dispatch. Multiple dispatch (or multi-methods, or
generic functions) is a particularly powerful form of object-oriented
programming that has been extensively studied in the past.
Most language designers and programmers seem to have concluded that
multiple dispatch is not really essential, and the feature is not often used
\cite{multipledispatch}. However, we have come to believe that
technical computing is its killer application.

Multiple dispatch appears at first to be about operator overloading:
defining the behavior of functions on various user-defined types.
We will show that it can do much more, given a few enhancements inspired
by dataflow type inference. This form of type inference is typically
applied to languages post-hoc, in attempts to add some static analysis
to languages not designed for it. Instead, our method signatures are
designed to exploit dataflow type inference, permitting definitions
that are easy to write, provide expressive power, and lead to
good performance. The result is a new tradeoff between performance
and flexiblity for technical computing.



%% \item Flexibility. The most important operations in technical computing
%% have a particularly uncooperative combination of characteristics:
%% they are highly polymorphic, the fine details of their behavior are
%% crucial to performance, and their binding time requirements
%% vary by application.
%% \end{enumerate}

%% These three characteristics are best explained through brief examples:

%% \begin{enumerate}
%% \item Polymorphism: array indexing accepts any number of arguments, each of
%% which may be of several different types.

%% \item Crucial details: the rank of an array must be statically known to fully
%% unroll index computations.

%% \item Binding time: in some applications array ranks are only known at run
%% time.
%% \end{enumerate}

%% Many other instances of these problems exist. In general, technical
%% computing is so diverse and exploratory that it is common for a language's
%% binding time decisions or type domains not to correspond to those
%% relevant to a particular technical application.


%% \subsection{Contributions}

%% - discusses a theory of what T.C. is, why its users do what they do

%% - suggests T.C. as the killer app for multiple dispatch

%% - demonstrates how to use it to implement library features previously
%% considered very advanced or out of reach



\section{Preliminary evaluation}

\begin{figure}
\begin{center}
\begin{tabular}{|l|r|r|r|}\hline
\textbf{Language} & \textbf{DR} & \textbf{CR} & \textbf{DoS} \\
\hline \hline
Gwydion    & 1.74 & 18.27 & 2.14 \\
\hline
OpenDylan  & 2.51 & 43.84 & 1.23 \\
\hline
CMUCL      & 2.03 &  6.34 & 1.17 \\
\hline
SBCL       & 2.37 & 26.57 & 1.11 \\
\hline
McCLIM     & 2.32 & 15.43 & 1.17 \\
\hline
Vortex     & 2.33 & 63.30 & 1.06 \\
\hline
Whirlwind  & 2.07 & 31.65 & 0.71 \\
\hline
NiceC      & 1.36 &  3.46 & 0.33 \\
\hline
LocStack   & 1.50 &  8.92 & 1.02 \\
\hline
Julia      & 5.86 & 51.44 & 1.54 \\
\hline
Julia Operators & 28.13 & 78.06 & 2.01 \\
\hline
\end{tabular}
\end{center}
\caption{
Comparison of Julia (1208 functions exported from the \texttt{Base} library)
to other languages with multiple dispatch, based on dispatch ratio (DR),
choice ratio (CR), and degree of specialization (DoS) \cite{multipledispatch}.
The ``Julia Operators'' row describes 47 functions with special syntax
(binary operators, indexing, and concatenation).
}
\label{dispatchratios}
\end{figure}

Together with the Julia open source community, we have developed a sizeable
library of technical computing functionality using our methodology. Some
simple statistics about the resulting definitions (Figure~\ref{dispatchratios})
reveal that something is indeed different here: multiple dispatch is
utilized much more fully in this domain. Community participation is
important, since it means that the code has many contributors, and
is considered reasonable and idiomatic by a significant sample of users
and developers.



\section{Related Work}

Type inference: \cite{kaplanullman, MLtypeinf, TICL,
pticl, nimble, taggingopt, typeinfjavascript}

Dataflow analysis: \cite{graphfree, abstractinterp, widening}

JIT compilation, dynamic language optimizations: \cite{futamura, pypyjit,
selflang, tracingjit, tracingjit2, druby,
profileguided, rubydust, lispcrit}

Optional and gradual typing: \cite{typedscheme, dylantypes, dyntype,
gradualobjects}

Multiple dispatch: \cite{closoverview, closspec, cecil, dieselspec, dylanlang}

Compiling technical computing languages: \cite{telescoping, telescopingvectorization,
slicehoisting, falcon, Rose:1999tt, Li:2013mf, matlabspecializer, majic}


\section{Schedule and Resources}

\begin{tabular}{|l|p{11cm}|}
\hline
\textbf{Date} & \textbf{Task} \\
\hline \hline
4/11/2014 & Paper submission (ACM SIGPLAN ARRAY '14) on aspects of the work \\
\hline
4/18/2014 & Thesis proposal submission \\
\hline
May 2014 & Outline and preliminary writing \\
\hline
June-July 2014 & Research and writing on historical background, analysis of use
cases \\
\hline
August 2014 & Identify case studies \\
\hline
September-October 2014 & Writing \\
\hline
11/9/2014 & Complete rough draft \\
\hline
1/9/2015 & Final draft \\
\hline
\end{tabular}

The only resources required for this work are access to a few computer systems,
and freely available software and web-based tools.

\newpage

\bibliographystyle{plain}
\bibliography{proposal}

\end{document}
